%
% Author: Nicola Bernardini <nicb@sme-ccppd.org>
%
% Copyright (c) 2006 Nicola Bernardini
%
% This work is licensed under the Creative Commons 
% Attribution-ShareAlike License. To 
% view a copy of this license, visit 
% http://creativecommons.org/licenses/by-sa/2.5/ 
% or send a letter to Creative Commons, 
% 559 Nathan Abbott Way, Stanford, California 94305, USA.
%
% Some rights reserved.
% SVNId : $Id: society.tex 347 2006-02-25 23:00:42Z nicb $
%
% Music Functions in our Societies
%
% Societies may support (or deny support to) different  music  functions
% according to their ideological needs:
% - promoted music
% - forbidden music
% - ignored music
% The usual mantra: nobody wants to listen to contemporary music. The
% porno/pulp/thriller comparison.
% Music device  transfer  between  functions  happen  when  they  become
% digested enough to perform the transfer: asymmetrical  rhythms,  noise
% ("post-digital music")
% Technology transfers happen when they fulfill some inner core of a given
% function: the example of electronics (speculative tool, then surrogate tool)
% Contradictions in terms: "cantanti  impegnati",
% "militant  composers"
% Funny  paradoxes:  contemporary   music   overproductivity, commercial
% singers "silences"
% A plea to contemporary musicology (?): let's not underestimate the power of
% functions
%
\svnInfo $Id: society.tex 347 2006-02-25 23:00:42Z nicb $
